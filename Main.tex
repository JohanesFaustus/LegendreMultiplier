\documentclass[twocolumn]{article}
\usepackage[a4paper, left=1cm, right=1cm, top=0cm, bottom=1cm]{geometry}
\usepackage{amsmath, amsthm, amssymb, esdiff, esint, indentfirst, longtable, array, dsfont, csquotes, fancyref}
\usepackage{graphicx}
\usepackage{titlesec}

\title{Determining the Largest Volume of a Paralelepiped Constrained Inside Ellipsoid by Lagrange Multipliers Method}
\author{Raffi C. Krisnanda}
\date{\today}

\begin{document}
\maketitle
\thispagestyle{empty}

\section{Introduction}
Consider ellipsoid described by the following equation:
\begin{equation}
    \frac{x^2}{a^2}+\frac{y^2}{b^2}+\frac{z^2}{c^2}=1
\end{equation}
What is the largest of volume of paralelepiped that can be inscribed inside this ellipsoid? This is an optimization problem that can be solved by Lagrange multiplier method.

\section{Lagrange Multipliers}
Let $f(x,y,z)$ be our function that we want to optimize and $\phi(x,y,z)=\text{const}$ be our constraint. We then set the total differential of $f(x,y,z)$ and $\phi(x,y,z)$ equal to zero
\begin{align}
    \frac{\partial f}{\partial x}dx + \frac{\partial f}{\partial y}dy + \frac{\partial f}{\partial y}dz&=0\\
    \frac{\partial \phi}{\partial x}dx + \frac{\partial \phi}{\partial y}dy + \frac{\partial \phi}{\partial y}dz&=0
\end{align}
Next, we construct the function
\begin{equation}
    F(x,y,z)=f(x,y,z)+\lambda\phi(x,y,z)
\end{equation}
and set its total derivative to zero
\begin{equation}
    \left(\frac{\partial f}{\partial x}+ \lambda\frac{\partial \phi}{\partial x}\right)dx + \left(\frac{\partial f}{\partial y}+ \lambda\frac{\partial \phi}{\partial y}\right)dy + \left(\frac{\partial f}{\partial z}+ \lambda\frac{\partial \phi}{\partial z}\right)dz = 0
\end{equation}
It follows that, for any value of $dx,\;dy,\;dz,$ we choose $\lambda$ such that 
\begin{equation}
    \frac{\partial f}{\partial x}+ \lambda\frac{\partial \phi}{\partial x}=0,\quad \frac{\partial f}{\partial y}+ \lambda\frac{\partial \phi}{\partial y}=0,\quad \frac{\partial f}{\partial z}+ \lambda\frac{\partial \phi}{\partial z}=0
\end{equation}
Putting it all together, to optimize $f(x,y,z)$ with constraint $\phi(x,y,z)$, we need to optimize $F(x,y,z)$, which obtained by solving three partial derivative equations and constraint equation $\phi(x,y,z)=\text{const}$. The equations in question are
\begin{align}
    \frac{\partial F}{\partial x}&=0\\
    \frac{\partial F}{\partial y}&=0\\
    \frac{\partial F}{\partial z}&=0\\
    \phi(x,y,z)&=\text{const.}
\end{align}

\section{Application}
The ellipsoid function acts as constraints, that left is to determine the function that we want to optimize. This requires some clever thinking. We begin by defining point $(x,y,z)$ be the corner of our parallelepiped. Now, this point is located in the first octant of our parallelepiped. The volume of this octant is 
\begin{equation}
    v=xyz
\end{equation}
Since the parallelepiped's sides are parallel the axis, its total volume is 
\begin{equation}
    V=8v
\end{equation}
Hence, the volume of our parallelepiped is 
\begin{equation}
    V=8xyz
\end{equation}
This is the function that we want to maximize. We then construct the function
\begin{equation}
    F(x,y,z)=8xyz+\lambda\left(\frac{x^2}{a^2}+\frac{y^2}{b^2}+\frac{z^2}{c^2}\right)
\end{equation}
The partial derivatives of $F$ read as 
\begin{align}
    \frac{\partial F}{\partial x}=8yz+\frac{2\lambda}{a^2}x,\quad
    \frac{\partial F}{\partial y}=8xz+\frac{2\lambda}{b^2}y,\quad
    \frac{\partial F}{\partial z}=8xy+\frac{2\lambda}{c^2}z
\end{align}
To find the maximum of $F$, we then must solve the partial derivative equations and constraint equation 
\begin{align}
    8yz+\dfrac{2\lambda}{a^2}x&=0\\
    8xz+\dfrac{2\lambda}{b^2}y&=0\\
    8xy+\frac{2\lambda}{c^2}z&=0\\
    \dfrac{x^2}{a^2}+\dfrac{y^2}{b^2}+\dfrac{z^2}{c^2}&=1
\end{align}
Multiplying the first equation by $x$, the second by $y$, the third by $z$ and adding them all together, we get 
\begin{equation}
    24xyz+2\lambda\left(\dfrac{x^2}{a^2}+\dfrac{y^2}{b^2}+\dfrac{z^2}{c^2}\right) = 24xyz+2\lambda=0
\end{equation}
Hence
\begin{equation}
    \lambda=-12xyz
\end{equation}
Substituting this into the partial derivative equation to obtain
\begin{align}
    8yz-\dfrac{24yz}{a^2}x^2&=0\implies x=\frac{\sqrt{3}}{3}a\\
    8xz-\dfrac{24xz}{b^2}y^2&=0\implies y=\frac{\sqrt{3}}{3}b\\
    8xy-\frac{24xy}{c^2}z^2&=0\implies z=\frac{\sqrt{3}}{3}c
\end{align}
Therefore, the maximum volume of said parallelepiped is 
\begin{equation}
    V=\frac{24\sqrt{3}}{27}abc
\end{equation}
\end{document}